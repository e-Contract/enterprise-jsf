% Enterprise JSF project.
%
% Copyright 2022 e-Contract.be BV. All rights reserved.
% e-Contract.be BV proprietary/confidential. Use is subject to license terms.

\chapter{Testing JSF Components}

\section{Selenium}
Using Selenium \cite{selenium} one can easily test web applications, and hence JSF components.
Let us test out the functionality of the following JSF page.
\lstinputlisting[language=XML]{testing/war/src/main/webapp/input-output.xhtml}
Notice that we gave every JSF component that we want to access from within our test an \texttt{id} attribute.

We represent this JSF page by the following test class.
\lstinputlisting[language=Java]{testing/tests/src/test/java/test/integ/be/e_contract/jsf/testing/InputOutputPage.java}
Notice how we locate the web elements using \texttt{By.id}.

Our basic integration test using Selenium looks as follows.
\lstinputlisting[language=Java]{testing/tests/src/test/java/test/integ/be/e_contract/jsf/testing/SeleniumTest.java}
Via \texttt{WebDriverManager} we make sure that the web driver \cite{webdriver} for the Chrome web browser is properly installed on our system.
Within the actual test we simply set some input value, perform a submit, and verify whether the output corresponds with the set input.

Since this Selenium based integration test assumes that the application has been deployed on a local running application server, we need some special constructs within our Maven POM file.
\lstinputlisting[language=XML]{testing/tests/pom.xml}
Notice that, per default, we disable the executing of the tests.
Only upon explicit activation of our \texttt{integration-tests} Maven profile,
the Selenium integration tests will run as part of a Maven build.

\subsection{PrimeFaces Selenium}

