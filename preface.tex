% Enterprise JSF project.
%
% Copyright 2022 e-Contract.be BV. All rights reserved.
% e-Contract.be BV proprietary/confidential. Use is subject to license terms.

\chapter*{Preface}
\addcontentsline{toc}{chapter}{Preface}
While there are a lot of good books/references on JSF, none I know of is completely dedicated to the development of custom JSF components.
With this effort, we try to create such "book/reference".

\begin{TODO}{Work in Progress}
	This "thing" is a work-in-progress, far from finished.
	It will probably never reach a "finished" state.
	For the moment we focus on providing sufficient source code examples.
	In a second phase, we will focus on adding some more meat to this document itself.
\end{TODO}

There is a strong tendency towards Single Page Applications (SPA), entirely constructed in some of the numerous Javascript frameworks out there.
However, for very large applications, where the (initial) development cycle surpasses the lifetime of the average Javascript framework, JSF is still a very viable solution.
By large applications we mean applications with at least 50 underlying data tables, hundreds of pages, and multiple dialogs per page.
Construction of very large applications also tend to go well beyond the scope of your favorite Javascript framework tutorial.
We are talking huge investments here for companies that are hence looking for solutions that have a proven lifetime, to prevent having to rework the application like every 5 years or so because the once chosen Javascript framework hit its EOL.

Instead of going in the direction of technical argumentation, let me take a step back and try to detail the bigger picture here.
JSF is far from perfect.
However, you rather want to feel proven pain, than finding yourself in a situation realizing after a few years of development that you made the wrong bet for the at-that-time popular front-end Javascript framework.
For huge applications, a company is not looking for fancy content to fill up the CVs of their freelancers with the latest kid on the block.
A company's primary goal here is to safeguard its heavy investment in terms of low-risk and hence low-cost maintenance over the long term.
This goal conflicts often with the goal of its freelancers: filling up CVs with fancy shit, and trying to monthly invoice for as long as possible.
Especially where there is a shortage within the market of good developers,
it is tempting for project managers/CTOs to give in to the demands of its development teams to go after the latest fancy Javascript framework.
What is also not helping here, is that the larger the required maintenance budget is for a project,
the more authority the project manager can have within the company,
which directly translates into more fine lunches with upper management.
Hence a lot of companies find themselves in the situation where they are stuck with unmaintainable maintenance costs on their large corner stone projects, and nobody has the decency to step forward and admit the stupidities made that burns money like hell.

I like JSF because it is not fancy.
It has been there for ages, and will be there for ages to come.
I find myself in the position where I own a company and hence need to keep the maintenance money burn under control.
I rather have difficulty finding developers, because of not fancy, than to find myself in a situation where I have to manage yet another new framework introduced by some dude that since has long left the building.
This is also one of the reasons why we still stick with Java 8 \cite{GoslingJoyEtAl14} and Java EE 8 \cite{JavaEE8} for the moment.

One technical argument however, I want to touch on here.
In my view, a component technology should be able to define both server-side and client-side behavior AT THE SAME TIME.
Hence, if I put a component,
\begin{lstlisting}[language=XML]
	<mycomps:myCoolComponent/>
\end{lstlisting}
on my web page, I want that this component can articulate behavior on both server-side and client-side, without the need to define some extras like a communication protocol or whatever.
As far as I know, all those fancy Javascript frameworks out there are completely missing the server-side part of the story, which forces you to start defining REST endpoints per server-side behavior you want to add for a certain component.

 \begin{flushright}
\today \\
Frank Cornelis 
\end{flushright}

We will be using the following boxes to draw special attention:
\begin{TODO}{Example TODO}
	Via these boxes, we indicate areas where we need to elaborate/fix things.
\end{TODO}
\begin{TIP}{Example Tip}
	Via these boxes, we give various tips.
\end{TIP}
\begin{ClownComputing}{Cloud computing versus clown computing}
	Via these boxes, we give strongly opinionated remarks.
	
	For example, cloud computing: you don't know where your application runs, where your data sits, how your data flows.
	The only thing you know for sure, is which VISA card is credited each month.  
\end{ClownComputing}