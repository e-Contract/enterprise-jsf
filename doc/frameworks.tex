% Enterprise JSF project.
%
% Copyright 2023-2024 e-Contract.be BV. All rights reserved.
% e-Contract.be BV proprietary/confidential. Use is subject to license terms.

\chapter{Front-end Frameworks}
Within this chapter we compare a popular Javascript front-end framework with JSF from a developer's perspective.
To do so, we construct a simple CRUD (create-read-update-delete) application using both technologies.
We are looking for a somehow objective reasoning whether Javascript based front-end frameworks are really that better compared to JSF web applications.
Some metrics to compare are:
\begin{itemize}
	\item Easy to maintain/understand build-system.
	\item Amount of framework API you need to have deep understanding of.
	\item Simplicity of the constructed application.
	\item Maintainability of the constructed application.
\end{itemize}
We construct the application using Jakarta EE 10 to be able to run it on the latest WildFly application server.


\section{The Model}
Both applications should use the same model.
Here we simply maintain a list of items.
The front-end should be able to display all items, and to remove/add items.
Our \texttt{Item} class looks as follows:
\lstinputlisting[language=Java]{../examples/frameworks/model/src/main/java/be/e_contract/model/Item.java}
Within a real production model our \texttt{Item} would of course be a JPA entity to manage its persistence.
But to keep things simple here, we avoid introducing JPA.
As you can see from the code, we use the \texttt{name} field somehow as a primary key.

Our \texttt{Model} class looks as follows:
\lstinputlisting[language=Java]{../examples/frameworks/model/src/main/java/be/e_contract/model/Model.java}
Again, to keep things simple we don't bother about aspects like transactionality, concurrency, security.
We do however introduce a "business" exception named \texttt{ExistingItem\allowbreak Exception} to be able to demonstrate proper handing of this within the front-end.
The source code of this exception looks as follows,
\lstinputlisting[language=Java]{../examples/frameworks/model/src/main/java/be/e_contract/model/ExistingItemException.java}

\section{JSF}
The JSF page looks as follows:
\lstinputlisting[language=XML]{../examples/frameworks/jsf/src/main/webapp/index.xhtml}
Notice the usage of our own Enterprise JSF tag library to ease opening and closing of the PrimeFaces dialogs.

The corresponding CDI controller looks as follows:
\lstinputlisting[language=Java]{../examples/frameworks/jsf/src/main/java/be/e_contract/jsf/Controller.java}
The implementation of our \texttt{@ViewScoped} CDI controller is straightforward.
During \texttt{@PostConstruct} we load the list of items from our model.
Our \texttt{add} and \texttt{removeSelected\allowbreak Item} JSF action methods do exactly as their names suggest.

The Maven POM file to build our project looks as follows:
\lstinputlisting[language=XML]{../examples/frameworks/jsf/pom.xml}

\section{Front-end Framework}
No matter which front-end framework we pick, we are forced to define a REST based API.
While we could do some ad-hoc JAX-RS API, we don't want to compromise on the "strong-typing" like we know from JSF.
We want that our server-side API is at all times in perfect sync with the client-side.
Hence this implies using OpenAPI \cite{openapi} and for the client-side a Typescript based framework.

The OpenAPI specification file looks as follows,
\lstinputlisting{../examples/frameworks/react/api/src/main/resources/openapi.yaml}
Important during the construction of the OpenAPI specification file is to pay special attention to the error handling.
While JSF inherently provides you with server-side based input validation, here you have to take care of the proper protocol definition regarding error handling yourself.

The Maven POM file to compile the OpenAPI towards JAX-RS interfaces looks as follows,
\lstinputlisting[language=XML]{../examples/frameworks/react/api/pom.xml}
While the \texttt{openapi-generator-maven-plugin} Maven plugin per default constructs an entire Maven project for you, I can hardly see why you would ever want that.
We only want the generated JAX-RS interfaces (and corresponding model), just like we did like 15 years ago with JAX-WS.
The Maven project is something we want to manage ourselves.
Hence via some configuration we instruct the \texttt{openapi-generator-maven-plugin} Maven plugin to limit itself to do just that.
Via some \texttt{build-helper-\allowbreak maven-plugin} magic we ensure that the generated JAX-RS interfaces actually also get compiled.
We also configure the \texttt{maven-jar-plugin} to exclude some generated classes to allow us to define the REST API context path at the application level instead.

The implementation of the generated OpenAPI JAX-RS interface looks as follows,
\lstinputlisting[language=Java]{../examples/frameworks/react/war/src/main/java/be/e_contract/react/ItemApiImpl.java}
Two things pop out here:
\begin{itemize}
	\item When looking at the \texttt{add} method implementation, we indeed are stuck with adding input validation manually.
	With JSF, this came for free.
	\item Within the \texttt{callList} method we have lovely message passing, mapping from one type to another.
	As Linus Torvalds once stated:
	\begin{quote}
		"message passing as the fundamental operation of the OS is just an exercise in computer science masturbation.
		It may feel good, but you don’t actually get anything DONE."
	\end{quote}
	In JSF, we could directly use our JPA entities within the front-end.
\end{itemize}

Our \texttt{web.xml} web deployment descriptor to activate JAX-RS looks as follows,
\lstinputlisting[language=XML]{../examples/frameworks/react/war/src/main/webapp/WEB-INF/web.xml}
We configure \texttt{resteasy.preferJacksonOverJsonB} to prevent a warning when deploying on WildFly.

We also need to allow CORS since the WildFly application server and the (React) development server will typically run on a different port.
We do this by means of the following \texttt{WEB-INF/undertow-handlers.conf} configuration file for the Undertow web server.
\lstinputlisting{../examples/frameworks/react/war/src/main/webapp/WEB-INF/undertow-handlers.conf}
Obviously within production, this configuration should get disabled.

\subsection{React}
As front-end framework we selected React.
Here we opted for a client-side rendered flavor, not a server-side rendered flavor with their so-called hydration simply because this is what JSF has been offering already for the last 20 years or so.

The build systems within the Javascript ecosystem are popping up like mushrooms.
However, since it seems like \texttt{esbuild} is for the moment in favor, we use this build system for our React application.
Central is the \texttt{package.json} file, which basically defines the dependencies of our project.
The \texttt{package.json} build file looks as follows:
\lstinputlisting[language=JSON]{../examples/frameworks/react/war/src/main/react/package.json}
It is simple and understandable.
On the one hand we have the \texttt{dependencies} for our React application itself.
And next to that we have the \texttt{devDependencies} for our build system.

The \texttt{esbuild} build script itself looks as follows:
\lstinputlisting[language=JavaScript]{../examples/frameworks/react/war/src/main/react/esbuild.config.mjs}
While you could invoke \texttt{esbuild} directly using its CLI,
we prefer using an ES Module script as this gives you maximum freedom on what to do.
We defined two modes:
\begin{itemize}
	\item \texttt{build} to simply build the project.
	\item \texttt{dev} to run the React application on a simple localhost dev server, with hot rebuilding enabled (watch).
\end{itemize}

The HTML file looks as follows,
\lstinputlisting[language=XML]{../examples/frameworks/react/war/src/main/react/public/index.html}
Via the \texttt{app} identifier on the \texttt{div} element, React will hook into our page at runtime.

As we don't want to loose the strong-typing as we know it from JSF, we opted for Typescript to define the React components.
The Typescript \texttt{index.tsx} file looks as follows,
\lstinputlisting{../examples/frameworks/react/war/src/main/react/src/index.tsx}
While using Typescript is initially a bit annoying, it quickly pays off as Visual Code can give better code completion hints.
It took a while to come up with a clean solution for the custom \texttt{AddItemDialog} and \texttt{RemoveItemDialog} using the \texttt{forwardRef} and \texttt{useImperativeHandle} construct.
It is often claimed that JSF has a steep learning curve, but once you start doing things beyond the React tutorials you're quickly faced with similar issues.
Due to the fast evolution of React, a lot of online blogs/documentation about React is also dated.

As for the UI components itself, similar to JSF where we default to PrimeFaces, for React we default to PrimeReact as we don't feel like reinventing components like a data table ourselves.
Hence from a point-of-view of look-and-feel and easy of styling, JSF is actually exactly as painful as React.
Whether you look into the JSF renderer class, or within the React JSX sections, you have to grasp the rendering if you want to apply some custom styling.

While there are probably some form based solutions out there to do the error handling within React, this is really the area where JSF is superior.
All this comes from the fact that JSF does the messaging for you behind the scenes whereas with React this aspect is like right in your face.

Compared with JSF, due to the React development server, the development cycle is drastically shorter \textbf{if} of course your REST endpoints do not require changes while developing the front-end.
But then again, runtimes like Quarkus also provide similar "hot deployment".
Keep in mind here that we went from one compilation step for JSF to three compilation steps: OpenAPI, REST API web application, and React compilation.

As for custom component development itself, this is difficult/dangerous to compare.
JSF components also cover the server-side behavior, whereas React components are limited to the client-side only.

Run the React application in development mode via:
\begin{lstlisting}[language=bash]
cd examples/frameworks/react/war/src/main/react
npm install
npm start
\end{lstlisting}

An argument that often pops up in the "JSF versus JS framework" discussions is that JSF does not scale well.
Indeed with JSF the view state is handled at the server-side whereas JS frameworks handle their state at the client-side.
However, unless your company is named Facebook or Google, I hardly see why this ever can be a valid argument.
Especially given the ever growing server-side CPU power and available memory, JSF can handle sufficient concurrent users way beyond most application's needs.

A way more important aspect for most companies is scalability in terms of added application functionality, let us call this functional scaling.
Does your framework scale well when adding another 50 data tables to your business model, accompanied with an additional 50 CRUD pages?
This is where, in my view, JS frameworks go into full mental breakdown.
By now all JS frameworks offer some kind of server-side rendering to be able to cope with functional scaling up to the point where they are basically replicating JSF.
The big difference here is that they drag along yet another NodeJS based server-side rendering runtime, which further complicates the management of application deployments.
Unless you are willing to rewrite your entire business logic in Typescript, with its corresponding callback hell because of async-is-better, you thus end up with two server-side runtimes: Java and NodeJS.
The big question here is to what benefit this serves besides the big cloud providers that try to cram as much paying customers on a given cloud infrastructure to keep the multi-billion golden cow alive for as long as possible.
Keep in mind that it is those very same big cloud providers that launched all of these JS frameworks in the first place.

In the meanwhile the poor average developer is fighting a technology stack with endless and ever increasing complexity, and to what purpose exactly?
To be able to put the latest new framework on their CV?
Where did we go wrong and gave in into this hype-driven development?
Is this maybe due to those same big cloud providers, that also happen to run the biggest social networks out there, that their constant plugging has killed of basic critical thinking?