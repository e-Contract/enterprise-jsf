% Enterprise JSF project.
%
% Copyright 2023 e-Contract.be BV. All rights reserved.
% e-Contract.be BV proprietary/confidential. Use is subject to license terms.

\chapter{Jakarta EE 10}

In 2017 Oracle handed over Java EE to the Eclipse Foundation.
Since Oracle wanted to protect the Java trademark, we had a renaming towards Jakarta EE.
The name refers to the largest city of the island of Java.

Jakarta EE 10 comes with a renaming of the \texttt{javax} package towards \texttt{jakarta}.
We also have similar renaming of the different XML namespaces.

As an example, our CDI controller targeting Jakarta EE 10 looks as follows:
\lstinputlisting[language=Java]{../examples/helloworld-ee10/war-ee10/src/main/java/be/e_contract/jsf/helloworld/InputOutputController.java}
The \texttt{jakarta} namespace prefix for the Java EE packages is the only change here.

Similarly, we have some namespace changes within our XHTML files as well.
\lstinputlisting[language=XML]{../examples/helloworld-ee10/war-ee10/src/main/webapp/index.xhtml}
Notice the \texttt{jakarta.faces} XML namespace prefix for the JSF namespaces.

The updated web descriptor looks as follows:
\lstinputlisting[language=XML]{../examples/helloworld-ee10/war-ee10/src/main/webapp/WEB-INF/web.xml}
Again \texttt{jakarta.faces} prefixes here.

The CDI descriptor becomes:
\lstinputlisting[language=XML]{../examples/helloworld-ee10/war-ee10/src/main/webapp/WEB-INF/beans.xml}
And the JSF descriptor becomes:
\lstinputlisting[language=XML]{../examples/helloworld-ee10/war-ee10/src/main/webapp/WEB-INF/faces-config.xml}

Jakarta EE 10 is supported by WildFly version 27 and higher.
This application server requires Java 11 as runtime.

While converting from Java EE 8 towards Jakarta EE 10 is not a big deal for a single project, things are different for (company-wide) JSF component libraries.
Since you will probably not be able to transition all your Java EE 8 projects at the same time, you need JSF component libraries that can support Java EE 8 in parallel with Jakarta EE 10.

Luckily this can easily be achieved using the \texttt{maven-shade-plugin} Maven plugin as demonstrated within the following Maven POM.
\lstinputlisting[language=XML]{../examples/helloworld-ee10/taglib/pom.xml}
Hence we keep producing a Java EE 8 JSF component library, and via the \texttt{maven-shade\allowbreak -plugin} Maven plugin we create a Jakarta EE 10 flavored version as well.
Notice the usage of the \texttt{jakarta} artifact classifier here.

Within our WAR POM we refer to the Jakarta EE 10 version of our JSF component library as follows:
\lstinputlisting[language=XML]{../examples/helloworld-ee10/war-ee10/pom.xml}
Via the \texttt{jakarta} \texttt{classifier} we include the Jakarta EE 10 flavor of our JSF component library.
