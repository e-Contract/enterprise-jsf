% Enterprise JSF project.
%
% Copyright 2022-2024 e-Contract.be BV. All rights reserved.
% e-Contract.be BV proprietary/confidential. Use is subject to license terms.

\chapter{Security Aspects of JSF Applications}
Within this chapter we want to elaborate on the security controls inherent to the JSF runtime, and the security controls to be put in place to be able to construct a secure application using JSF.
First of all, let us define what a secure application is.
Let me set the stage here very clear: there exists no such thing as a secure application.
Throw enough money and time at something, and you can hack it.
Always.
This is caused by the inherent altruistic nature of developers wanting to construct something useful to others, and thinking that all people go with this flow.
Unfortunately the world doesn't tick this way.
There are only assholes out there, and next to that, entire teams of even bigger assholes.
An asshole will hack your application if the benefit exceeds the effort.
It's as simple as that.
With this in mind, we can define an application as being "secure" if the effort of hacking it exceeds the value of the information it is supposed to protect.
An asshole, as being a hole itself, searches for other holes.
Hence what are the potential holes within your application?
Organizations like OWASP already did a good job at defining this by means of their Top 10 vulnerability list.
So let us translate this list to JSF parlance.

\begin{TODO}{TODO}
	Write me.
\end{TODO}



