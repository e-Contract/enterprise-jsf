% Enterprise JSF project.
%
% Copyright 2022-2024 e-Contract.be BV. All rights reserved.
% e-Contract.be BV proprietary/confidential. Use is subject to license terms.

\chapter{JSF runtimes}
Within this chapter, we take a look at different runtime environments capable of hosting JSF based web applications.
First we take a look at different JSF implementations.
Next we investigate how to configure these JSF implementations within different runtime environments.

\section{JSF implementations}
There are two JSF implementations available.
\subsection{Mojarra}
This is the JSF reference implementation \cite{mojarra}.
\subsection{Apache MyFaces}
This is a JSF implementation provided by the Apache Software Foundation \cite{myfaces}.

\section{Alternative Runtime Environments}
Next to Java/Jakarta EE application servers,
one can also target other runtimes like Spring or Quarkus.
Interesting here is to highlight the differences between Java/Jakarta EE runtimes and those alternatives.
While the development pace and feature sets of runtimes like Spring and Quarkus seems to be better compared to a classical Java/Jakarta EE runtime, one should be careful here as to where you are about to jump into.
Java/Jakarta EE comes with some very important characteristics:
\begin{itemize}
	\item There is a disconnect between the specification on the one hand, and the underlying implementation on the other hand.
	\item Each specification has been reached in consensus between 10 or more participants of different vendors.
	This results in specifications where people really though about it, tried things out, had discussions about.
	A side-effect of this is that Java/Jakarta EE specifications tend to progress slower.
	This is a good thing.
	You don't want half baked specifications and half baked implementations on which your application will depend.
	\item Most Java/Jakarta EE specifications have multiple independent implementations by different vendors.
	Each of those implementations are tested via some common TCK (Technology Compatibility Kit) to verify conformity with the corresponding specification.
	\item The different Java EE specification are aligned with each other.
	For example, there is CDI as a spec, and also how CDI is enabled within Servlet containers, JSF runtimes, etc.
	\item The most important one maybe, the Java EE top-level specification defines a set of underlying specs which an application server implementation should provide.
	Hence you don't have to start selecting different layers/extensions, whatever they might call it, yourself with the risk of runtime incompatibilities.
	While the YAGNI (You aren't gonna need it) principle should be applied to your application itself, there is absolutely no reason to try to apply it to the selected application runtime at all.
	Most business applications in the end will use almost every Java EE feature available, so you might as well directly start with a fully featured Java EE runtime.
\end{itemize}
All of this results in very mature specifications and well tested implementations.
Hence it gives you stable application runtimes.
This is especially true if you stick with a "current Java/Jakarta EE version minus one" strategy for production deployment as it comes to the selected application server.
This gives you an application runtime that had sufficient time to harden out,
allowing you to focus on implementing your business use cases instead of chasing after bugs in somebody else's code (or getting stuck in the support contract treadmill).
You like to find yourself at the leading edge, not the bleeding edge here.

The major disadvantage of course is that your developers cannot put the latest kid on the block on their CV.
The easiest way to tackle this,
is to tell them that they can go and choose for a "more modern" runtime if they also commit to waking up at 3 o'clock in the night in case that the application runtime explodes.
Been there, done that, no thanks.
Most of them end the discussion at that point.
People like to sleep at night, including CTO's and CEO's.

Of course there are clear advantages to alternative runtimes like Spring or Quarkus.
But as usual an objective rational argumentation should be found.
Just because it is "cool" or "everybody else is using it" can never be a sufficient argument.

Within the next sections, we demonstrate Spring and Quarkus as alternative application runtimes.

\subsection{Spring}

Maven project file:
\lstinputlisting[language=XML]{../examples/spring/pom.xml}

Configuration file \texttt{src/main/resources/application.properties}:
\lstinputlisting{../examples/spring/src/main/resources/application.properties}

Our \texttt{Application.java} entry class:
\lstinputlisting[language=Java]{../examples/spring/src/main/java/be/e_contract/ejsf/spring/Application.java}

Index page \texttt{src/main/resources/META-INF/resources/index.xhtml}:
\lstinputlisting[language=XML]{../examples/spring/src/main/webapp/index.xhtml}

Our \texttt{HelloWorldBean.java} controller:
\lstinputlisting[language=Java]{../examples/spring/src/main/java/be/e_contract/ejsf/spring/HelloWorldBean.java}

A \texttt{RedirectToIndexController.java} controller that will redirect requests on \texttt{/} to \texttt{index.xhtml}:
\lstinputlisting[language=Java]{../examples/spring/src/main/java/be/e_contract/ejsf/spring/RedirectToIndexController.java}

\begin{TIP}{Java Runtime}
	Please notice that Spring requires at least Java version 11 as runtime.
\end{TIP}

Compile and run the application via:
\begin{lstlisting}[language=bash]
	cd examples/spring
	mvn clean spring-boot:run
\end{lstlisting}
The application is available at:
\url{https://localhost:8080/test}



\subsection{Quarkus}

Compared to a classical Java EE application server like WildFly \cite{WildFly}, the Quarkus \cite{quarkus} platform offers some advantages.
\begin{itemize}
	\item Most important advantage is the live coding feature, where edited Java classes automatically get recompiled and reloaded at runtime.
	This feature cuts down the development-deploy-test cycle tremendously.
	\item Easy configuration of the runtime.
\end{itemize}
Besides the obvious advantages, which you can find all over the internet, there might also be some disadvantages to such more recent platforms.
\begin{itemize}
	\item You have to construct yourself a platform, by activating different so called extensions.
	Whether your selected extensions will not fight each other at runtime is entirely your problem.
	With a Java EE application server, you don't have to worry about this aspect.
	The application server vendor had this fight for you.
	\item Quarkus is still very young compared to the established Java EE application servers.
	Hence keeping your application up to date with the latest version of Quarkus might eat away some of your time.
	\item A classical application server has a stronger focus on production stability and security versus adding new features all the time to keep up the fight with Spring Boot \cite{spring}.
\end{itemize}

The Maven project file for our example Quarkus application looks as follows.
\lstinputlisting[language=XML]{../examples/quarkus/pom.xml}
We simplified the POM file compared to what Quarkus normally generates for you.
This was done to more easily highlight the required configuration aspects.
Basically all the heavy lifting to build the Quarkus based runtime is handled by the \texttt{quarkus-maven-plugin} Maven plugin.
We use the \texttt{myfaces-quarkus} Quarkus extension to provide us with a JSF runtime based on Apache MyFaces \cite{myfaces}.


Via the \texttt{src/main/resources/application.properties}  Quarkus application configuration file we can configure different aspects of the runtime. For example:
\lstinputlisting{../examples/quarkus/src/main/resources/application.properties}

The web deployment descriptor \texttt{src/main/resources/META-INF/web.xml} looks as follows:
\lstinputlisting[language=XML]{../examples/quarkus/src/main/resources/META-INF/web.xml}
As you can see, we no longer have to configure the JSF servlet ourselves.
This has been taken care of by the \texttt{myfaces-quarkus} extension.

Our example index page \texttt{src/main/resources/META-INF/resources/index.xhtml} looks as follows:
\lstinputlisting[language=XML]{../examples/quarkus/src/main/resources/META-INF/resources/index.xhtml}

Our \texttt{HelloController.java} CDI based controller:
\lstinputlisting[language=Java]{../examples/quarkus/src/main/java/be/e_contract/ejsf/quarkus/HelloController.java}
As you can see, the classical Java EE API's can be used directly within Quarkus, which allows for an easy transition to this new runtime.

\begin{TIP}{Java Runtime}
	Quarkus requires at least Java version 11 as runtime.
\end{TIP}

Compile and run the application via:
\begin{lstlisting}[language=bash]
	cd examples/quarkus
	mvn clean compile quarkus:dev
\end{lstlisting}
Press "w" to view the application in a web browser.

Create a Quarkus image via:
\begin{lstlisting}[language=bash]
	cd examples/quarkus
	mvn clean package
\end{lstlisting}
Run the Quarkus image via:
\begin{lstlisting}[language=bash]
	cd examples/quarkus/target/quarkus-app
	java -jar quarkus-run.jar
\end{lstlisting}
Create a Quarkus native image using a Docker container via:
\begin{lstlisting}[language=bash]
	cd examples/quarkus
	mvn clean package -Dquarkus.package.type=native -Dquarkus.native.container-build=true
\end{lstlisting}
\begin{TODO}{Native Image}
The above does not work for somehow reason.
\end{TODO}

\subsubsection{Tag library}
Since the default setup of a Quarkus project is to have a single Maven module,
embedding a tag library has to be done by simply adding the tag library code and descriptors within the same Maven project as your application itself.


