% Enterprise JSF project.
%
% Copyright 2022-2023 e-Contract.be BV. All rights reserved.
% e-Contract.be BV proprietary/confidential. Use is subject to license terms.

\chapter{JSF runtimes}

\section{JSF implementations}

\section{Alternative Runtime Environments}

\subsection{Spring}

Maven project file:
\lstinputlisting[language=XML]{../spring/pom.xml}

Configuration file \texttt{src/main/resources/application.properties}:
\lstinputlisting{../spring/src/main/resources/application.properties}

Our \texttt{Application.java} entry class:
\lstinputlisting[language=Java]{../spring/src/main/java/be/e_contract/ejsf/spring/Application.java}

Index page \texttt{src/main/resources/META-INF/resources/index.xhtml}:
\lstinputlisting[language=XML]{../spring/src/main/webapp/index.xhtml}

Our \texttt{HelloWorldBean.java} controller:
\lstinputlisting[language=Java]{../spring/src/main/java/be/e_contract/ejsf/spring/HelloWorldBean.java}

A \texttt{RedirectToIndexController.java} controller that will redirect requests on \texttt{/} to \texttt{index.xhtml}:
\lstinputlisting[language=Java]{../spring/src/main/java/be/e_contract/ejsf/spring/RedirectToIndexController.java}

\begin{TIP}{Java Runtime}
	Please notice that Spring requires at least Java version 11 as runtime.
\end{TIP}

Compile and run the application via:
\begin{lstlisting}[language=bash]
	cd spring
	mvn clean spring-boot:run
\end{lstlisting}
The application should now be available at:
https://localhost:8080/test



\subsection{Quarkus}

Compared to a classical Java EE application server like WildFly \cite{WildFly}, the Quarkus \cite{quarkus} platform offers some advantages.
\begin{itemize}
	\item Most important advantage is the live coding feature, where edited Java classes automatically get recompiled and loaded at runtime.
	This feature cuts down the development-deploy-test cycle tremendously.
	\item Easy configuration of the runtime.
\end{itemize}
Besides the obvious advantages, which you can find all over the internet, there might also be some disadvantages to such more recent platforms.
\begin{itemize}
	\item You have to construct yourself a platform, by activating different so called extensions.
	Whether your selected extensions will not fight each other at runtime is entirely your problem.
	With a Java EE application server, you don't have to worry about this aspect.
	The application server vendor had this fight for you.
	\item Quarkus is still very young compared to the established Java EE application servers.
	Hence keeping your application up to date with the latest version of Quarkus might eat away some of your time.
	\item A classical application server has a stronger focus on production stability versus adding new features all the time to keep up the fight with Spring Boot \cite{spring}.
\end{itemize}

Maven project file:
\lstinputlisting[language=XML]{../quarkus/pom.xml}
Notice here that we simplified our POM file compared to what Quarkus normally generates for you.
This was done to more easily highlight to required configuration aspects.

Configuration file \texttt{src/main/resources/application.properties}:
\lstinputlisting{../quarkus/src/main/resources/application.properties}

Web deployment descriptor \texttt{src/main/resources/META-INF/web.xml}:
\lstinputlisting[language=XML]{../quarkus/src/main/resources/META-INF/web.xml}

Index page \texttt{src/main/resources/META-INF/resources/index.xhtml}:
\lstinputlisting[language=XML]{../quarkus/src/main/resources/META-INF/resources/index.xhtml}

Our \texttt{HelloController.java} CDI based controller:
\lstinputlisting[language=Java]{../quarkus/src/main/java/be/e_contract/ejsf/quarkus/HelloController.java}

\begin{TIP}{Java Runtime}
	Please notice that Quarkus requires at least Java version 11 as runtime.
\end{TIP}

Compile and run the application via:
\begin{lstlisting}[language=bash]
	cd quarkus
	mvn clean compile quarkus:dev
\end{lstlisting}
Press "w" to view the application in a web browser.

\subsubsection{Tag library}
Since the default setup of a Quarkus project is to have a single Maven module,
embedding a tag library has to be done by simply adding the tag library code and descriptors within the same Maven project as your application itself.


