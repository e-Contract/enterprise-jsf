% Enterprise JSF project.
%
% Copyright 2022-2023 e-Contract.be BV. All rights reserved.
% e-Contract.be BV proprietary/confidential. Use is subject to license terms.

\chapter{PrimeFaces}
\label{chap:primefaces}
JSF has known a lot of component libraries over its history.
After almost two decades of JSF however, natural selection did set in,
resulting in the survival of just a few really high quality JSF component libraries.
The most prominent here is PrimeFaces \cite{PrimeFaces}.
Some of its major features are:
\begin{itemize}
	\item Lots of components well beyond the standard JSF component set.
	\item Client-side API for the components, called widgets.
	\item Dialog framework.
	\item Configurable style.
	\item Responsiveness.
\end{itemize}
We want to highlight the importance of JSF component libraries like PrimeFaces.
Constructing a web application using vanilla JSF is just completely nuts.
The basic JSF runtime is just that, very basic.
You don't want to reinvent common components like lazy loaded paginated tables yourself.
This has already been done by people (at PrimeTek, the company behind PrimeFaces) that are way more experienced at doing so than you can ever achieve.
Instead, focus on solving the business problems of your customers.

Same remarks apply to other frond-end frameworks as well, like Angular, React, or Vue, just to name a few.
No matter which framework you are using: \textbf{development of high quality custom components takes a lot of effort}.
What makes component development so challenging is that you operate at the intersection of different technologies: JSF component API, HTML, CSS, Javascript, PrimeFaces Widget API, jQuery, and the (business) requirements imposed on your component.
Within each of these domains you need an almost expert level to be able to achieve something truly generic and usable.
This is hard, no matter which framework that you target.
Hence if you do so, isolate these components within dedicated libraries so multiple projects within your organization can benefit from it, making worth the investment.


\section{Introduction}
Let's start off by doing some basic "hello world" using PrimeFaces.
Converting our \nameref{sec:basic-input-output} example from Section \ref{sec:basic-input-output}  on page \pageref{sec:basic-input-output} to PrimeFaces gives us the following result.
\lstinputlisting[language=XML]{../examples/helloworld-primefaces/war/src/main/webapp/hello-world.xhtml}
The most visible change here is that they renamed the \texttt{render} attribute to the more intuitive \texttt{update} attribute.
Where with vanilla JSF you have to explicitly indicate that you want to perform an AJAX call, with PrimeFaces components this is the default.
Hence \texttt{p:commandButton} will perform an AJAX call per default, voiding the need for an explicit \texttt{p:ajax} tag.
You can disable this behavior by means of setting the \texttt{ajax="false"} attribute on the \texttt{p:commandButton} component.

\section{CRUD with Dialogs}
PrimeFaces allows you to construct dialogs in a very intuitive manner, yielding an SPA alike experience towards end-users.
Within this section, we demonstrate the dialog framework provided by PrimeFaces.
Our CRUD (create-read-update-delete) JSF example page looks as follows:
\lstinputlisting[language=XML]{../examples/helloworld-primefaces/war/src/main/webapp/crud-dialogs.xhtml}
A dialog is declared using the \texttt{p:dialog} PrimeFaces component.
Notice how to invoke the client-side API of certain widgets via the \texttt{widgetVar} attributes to give them a name, and via the \texttt{PF} function to locate these widgets.

The Javascript resource \texttt{crud-dialogs.js} contains the following:
\lstinputlisting[language=Javascript]{../examples/helloworld-primefaces/war/src/main/webapp/resources/js/crud-dialogs.js}
The \texttt{itemAdded} AJAX callback argument, available via \texttt{args}, is set within the CDI controller as follows.

\lstinputlisting[language=Java]{../examples/helloworld-primefaces/war/src/main/java/be/e_contract/jsf/helloworld/CRUDController.java}
Via \texttt{addCallbackParam} we can add parameters to the AJAX response that are picked up by the client-side Javascript.

\section{Close Dialog Client Behavior}
Within this section we demonstrate how to define a JSF client behavior tailored for the PrimeFaces component library.
Let us construct a custom client behavior to close dialogs.
This client behavior can be attached to for example \texttt{p:commandButton} components within dialogs.
The source code of our client behavior implementation looks as follows.
\lstinputlisting[language=Java]{../examples/helloworld-primefaces/taglib/src/main/java/be/e_contract/jsf/taglib/CloseDialogClientBehavior.java}
Here we locate the parent dialog of the \texttt{p:commandButton} component.
Once we have located the dialog, we use its \texttt{widgetVar} to be able to close it at the client-side.

We register this custom component within our \texttt{.taglib.xml} tag library descriptor as follows.
\lstinputlisting[language=XML,linerange=closeDialog]{../examples/helloworld-primefaces/taglib/src/main/resources/META-INF/ejsf.taglib.xml}

We use this custom client behavior as follows within PrimeFaces dialogs:
\lstinputlisting[language=XML]{../examples/helloworld-primefaces/war/src/main/webapp/close-dialog.xhtml}
Clicking the "Dismiss" \texttt{p:commandButton} within the dialog will trigger the \texttt{closeDialog} client-side behavior and close the corresponding \texttt{p:dialog}.
Notice the usage of \texttt{return false;} on the \texttt{onclick} client-side event to prevent further propagation towards an actual AJAX call.

\section{Widgets}
Within this section, we construct a component with a client-side API, called a widget in PrimeFaces parlance.
The PrimeFaces Javascript frameworking basically provides us with the following:
\begin{itemize}
	\item A mapping from a widget variable name towards the widget Javascript instance.
	This allows us to invoke Javascript methods on our client-side Javascript widget.
	\item Lifecycle calbacks towards our widgets so we can keep track of AJAX updates and such.
\end{itemize}
Our custom component with corresponding client-side widget is composed out of the actual component, and the corresponding renderer.
The source code for the component looks as follows.
\lstinputlisting[language=Java]{../examples/helloworld-primefaces/taglib/src/main/java/be/e_contract/jsf/taglib/ExampleComponent.java}
The component defines a \texttt{widgetVar} attribute that is used by our custom renderer.
The different \texttt{@ResourceDependency} annotation declarations are required to provide the necessary client-side runtime for our custom widget.
As you see, PrimeFaces fully embraced jQuery \cite{jQuery} on the client-side.
Hence a good understanding of jQuery is of eminent importance when developing PrimeFaces widget components.

The source code for the corresponding renderer looks as follows.
\lstinputlisting[language=Java]{../examples/helloworld-primefaces/taglib/src/main/java/be/e_contract/jsf/taglib/ExampleRenderer.java}
Notice here that we extend the PrimeFaces \texttt{CoreRenderer} class.
Using the PrimeFaces \texttt{WidgetBuilder} we bind our client-side widget to our component.
We can pass attributes from server-side to client-side via the \texttt{WidgetBuilder} \texttt{attr} method.
The \texttt{finish} method takes care of the necessary inline \texttt{script} rendering to activate our widget at the client-side.

The source code of the Javascript resource \texttt{example-widget.js} looks as follows.
\lstinputlisting[language=Javascript]{../examples/helloworld-primefaces/taglib/src/main/resources/META-INF/resources/ejsf/example-widget.js}
Our widget has to extend the PrimeFaces Javascript \texttt{BaseWidget} class like thingy.
Via the \texttt{init} method our widget gets initialized by the PrimeFaces client-side framework.
Via the \texttt{cfg} parameter we can access our custom attributes.
Our widget provides two methods \texttt{getValue} and \texttt{setValue} that do exactly as they are named.
We also override the \texttt{destroy} and \texttt{refresh} methods to demonstrate the lifecycle of widgets.
The PrimeFaces Javascript frameworking will invoke \texttt{refresh} when a widget component gets updated during an AJAX call.
In this case the existing Javascript widget gets rebased towards the update component DOM element.
The \texttt{destroy} method gets invoked when the corresponding widget component gets removed (i.e., \texttt{rendered} is \texttt{false}) as part of an AJAX call.
When overriding these methods, it is important to keep invoking the \texttt{\_super} methods as these contain listener logic.

We register this custom component and its corresponding renderer within our \texttt{.taglib\allowbreak .xml} tag library descriptor as follows.
\lstinputlisting[language=XML,linerange=exampleComponent]{../examples/helloworld-primefaces/taglib/src/main/resources/META-INF/ejsf.taglib.xml}

An example usage is demonstrated below:
\lstinputlisting[language=XML]{../examples/helloworld-primefaces/war/src/main/webapp/widget.xhtml}
Via the \texttt{PF} function we locate our client-side widget corresponding with the \texttt{example\allowbreak Component}.
On this widget, we can invoke the different defined methods like \texttt{setValue} as we demonstrate within the example.
When unchecking the "Rendered" checkbox and clicking the "Update" button, you will notice that the \texttt{destroy} method on the client-side widget gets invoked.

The corresponding CDI controller is given below.
\lstinputlisting[language=Java]{../examples/helloworld-primefaces/war/src/main/java/be/e_contract/jsf/helloworld/WidgetController.java}


\section{Turning a Javascript library into a JSF component}
Within this section we demonstrate how to turn a Javascript library into an easy to use JSF component.
We wrap the Apache ECharts \cite{ECharts} Javascript library within a JSF component.
The Apache ECharts library allows you to generate nice looking charts via a Javascript/JSON API on the client-side.
The source code for the component looks as follows.
\lstinputlisting[language=Java]{../examples/helloworld-primefaces/taglib/src/main/java/be/e_contract/jsf/taglib/EChartsComponent.java}
Notice how we include the ECharts \texttt{echarts.min.js} Javascript library via the \texttt{@Resource\allowbreak Dependency} annotation.

The source code for the corresponding renderer is given below.
\lstinputlisting[language=Java]{../examples/helloworld-primefaces/taglib/src/main/java/be/e_contract/jsf/taglib/EChartsRenderer.java}
The value is not directly passed during \texttt{encodeBegin} as we normally do.
Instead we let the widget fetch the value asynchronously via an AJAX call.
Within the \texttt{decode} method we pass the component value towards the client-side via \texttt{addCallbackParam}.

The source code of the Javascript resource \texttt{echarts-widget.js} looks as follows.
\lstinputlisting[language=Javascript]{../examples/helloworld-primefaces/taglib/src/main/resources/META-INF/resources/ejsf/echarts-widget.js}
We perform a JSF AJAX call to retrieve our component value, and use this to configure the ECharts Javascript instance within the \texttt{oncomplete} method.

We register this custom component and its corresponding renderer within our \texttt{.taglib\allowbreak .xml} tag library descriptor as follows.
\lstinputlisting[language=XML,linerange=echarts]{../examples/helloworld-primefaces/taglib/src/main/resources/META-INF/ejsf.taglib.xml}

An example usage is demonstrated below:
\lstinputlisting[language=XML]{../examples/helloworld-primefaces/war/src/main/webapp/echarts.xhtml}

With the CDI controller as follows:
\lstinputlisting[language=Java]{../examples/helloworld-primefaces/war/src/main/java/be/e_contract/jsf/helloworld/EChartsController.java}

To bundle the Javascript code of Apache EChart within our JSF library, we use some specific Maven build instructions as indicated in the following \texttt{pom.xml}.
\lstinputlisting[language=XML]{../examples/helloworld-primefaces/taglib/pom.xml}
With the \texttt{src/main/node/package.json} file containing:
\lstinputlisting[language=JSON]{../examples/helloworld-primefaces/taglib/src/main/node/package.json}
Hence we use \texttt{node} and \texttt{npm} to retrieve the ECharts Javascript package.

Via the \texttt{maven-resources-plugin} we selectively pick what we need out of the downloaded \texttt{echarts} node module as was retrieved by the \texttt{frontend-maven-plugin} Maven plugin.

While in this example we bundled the external resources from \texttt{echarts} directly within the JSF tag library itself,
it is strongly advised to separate such external resources within a dedicated JAR.
This strategy brings several advantages:
\begin{itemize}
	\item obviously a smaller tag library JAR, and hence your WARs/EARs remain as small as possible, safeguarding build and deployment time.
	\item because all \texttt{npm} stuff lives outside the tag library, the development-deployment cycle improves drastically for the component development itself.
	\item cleaner tag library \texttt{pom.xml}.
\end{itemize}