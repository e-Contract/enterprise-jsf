% Enterprise JSF project.
%
% Copyright 2022-2023 e-Contract.be BV. All rights reserved.
% e-Contract.be BV proprietary/confidential. Use is subject to license terms.

\chapter{PrimeFaces}
\label{chap:primefaces}
JSF has known a lot of component libraries over its history.
After almost two decades of JSF however, natural selection did set in,
resulting in the survival of just a few really high quality JSF component libraries.
The most prominent here is PrimeFaces \cite{PrimeFaces}.
Some of its major features are:
\begin{itemize}
	\item Lots of components well beyond the standard JSF component set.
	\item Client-side API for the components, called widgets.
	\item Dialog framework.
	\item Configurable style.
	\item Responsiveness.
\end{itemize}
We want to highlight the importance here of JSF component libraries like PrimeFaces.
Constructing a web application using vanilla JSF is just completely nuts.
The basic JSF runtime is just that, very basic.
You don't want to reinvent common components like lazy loaded paginated tables yourself.
This has already been done by people (at PrimeTek, the company behind PrimeFaces) that are way more experienced at doing so than you can ever achieve.
Instead, focus on solving the business problems of your customers.

Same remarks apply to other frond-end frameworks as well, like Angular, React, or Vue, just to name a few.
Nomatter which framework you are using: \textbf{development of high quality custom components takes a lot of effort}.
Hence if you do so, isolate these components within dedicated libraries so multiple projects within your organization can benefit, making worth the investment.


\section{Introduction}
Converting our \nameref{sec:basic-input-output} example from Section \ref{sec:basic-input-output}  on page \pageref{sec:basic-input-output} to PrimeFaces gives us the following result.
\lstinputlisting[language=XML]{../helloworld-primefaces/war/src/main/webapp/hello-world.xhtml}
Most visible change here is that they renamed the \texttt{render} attribute to the more intuitive \texttt{update} attribute.
Where with vanilla JSF you have to explicitly indicate that you want to perform an AJAX call, with PrimeFaces this is the default.
Hence \texttt{p:commandButton} will perform an AJAX call per default, voiding the need for an explicit \texttt{p:ajax} tag.
You can disable this behavior by means of setting \texttt{ajax="false"}.

\section{CRUD with Dialogs}
PrimeFaces allows you to construct dialogs in a very intuitive manner.
Within this section, we demonstrate the dialog framework provided by PrimeFaces.
Our JSF example page looks as follows:
\lstinputlisting[language=XML]{../helloworld-primefaces/war/src/main/webapp/crud-dialogs.xhtml}
Notice how to invoke the client-side API of certain widgets via the \texttt{widgetVar} attributes to give them a name, and via the \texttt{PF} function to locate these widgets.

The Javascript resource \texttt{crud-dialogs.js} contains the following:
\lstinputlisting[language=Javascript]{../helloworld-primefaces/war/src/main/webapp/resources/js/crud-dialogs.js}
The \texttt{itemAdded} AJAX callback argument, available via \texttt{args}, is set within the CDI controller below.

\lstinputlisting[language=Java]{../helloworld-primefaces/war/src/main/java/be/e_contract/jsf/helloworld/CRUDController.java}
Via \texttt{addCallbackParam} we can add parameters to the Ajax response that can be picked up by the client-side Javascript widget.

\section{Close Dialog Client Behavior}
Within this section we demonstrate how to define a JSF client behavior tailored for the PrimeFaces component library.
Let us construct a custom client behavior to close dialogs.
This client behavior can be attached to for example \texttt{p:commandButton} components.
The source code of our client behavior implementation is given below.
\lstinputlisting[language=Java]{../helloworld-primefaces/taglib/src/main/java/be/e_contract/jsf/taglib/CloseDialogClientBehavior.java}
Here we locate the dialog in which the \texttt{p:commandButton} sits.
Once we have located the dialog, we use its \texttt{widgetVar} to be able to close it at the client-side.

We register this custom component within our \texttt{.taglib.xml} tag library descriptor.
\lstinputlisting[language=XML,linerange=closeDialog]{../helloworld-primefaces/taglib/src/main/resources/META-INF/ejsf.taglib.xml}

We can now use this custom client behavior as follows within PrimeFaces dialogs:
\lstinputlisting[language=XML]{../helloworld-primefaces/war/src/main/webapp/close-dialog.xhtml}
Clicking the "Dismiss" \texttt{p:commandButton} will trigger the \texttt{closeDialog} client-side behavior and close the corresponding \texttt{p:dialog}.

\section{Widgets}
Within this section, we construct a component with a client-side API, called a widget in PrimeFaces parlance.
Our custom component with corresponding client-side widget is composed out of the actual component, and the corresponding renderer.
The source code for the component is given below.
\lstinputlisting[language=Java]{../helloworld-primefaces/taglib/src/main/java/be/e_contract/jsf/taglib/ExampleComponent.java}
The component defines a \texttt{widgetVar} attribute that is used by our custom renderer.
The different \texttt{@ResourceDependency} annotation declarations are required to provide the necessary client-side runtime for our custom widget.

The source code for the corresponding renderer is given below.
\lstinputlisting[language=Java]{../helloworld-primefaces/taglib/src/main/java/be/e_contract/jsf/taglib/ExampleRenderer.java}
Using the PrimeFaces \texttt{WidgetBuilder} we bind our client-side widget to our component.
We can pass attributes from server-side to client-side via the \texttt{WidgetBuilder} \texttt{attr} method.
The \texttt{finish} method takes care of the necessary rendering to activate our widget.

The source code of the Javascript resource \texttt{example-widget.js} is given below.
\lstinputlisting[language=Javascript]{../helloworld-primefaces/taglib/src/main/resources/META-INF/resources/ejsf/example-widget.js}
Our widget has to extend the PrimeFaces \texttt{BaseWidget}.
Via the \texttt{init} method our widget gets initialized by the PrimeFaces client-side framework.

We register this custom component within our \texttt{.taglib.xml} tag library descriptor.
\lstinputlisting[language=XML,linerange=exampleComponent]{../helloworld-primefaces/taglib/src/main/resources/META-INF/ejsf.taglib.xml}

An example usage is demonstrated below:
\lstinputlisting[language=XML]{../helloworld-primefaces/war/src/main/webapp/widget.xhtml}
Via the \texttt{PF} function we can locate our client-side widget corresponding with the \texttt{exampleComponent}.
On this widget, we can invoke the different defined methods like \texttt{setValue}.

\section{Turning a Javascript library into a JSF component}
Within this section we demonstrate how to turn a Javascript library into an easy to use JSF component.
We wrap the Apache ECharts \cite{ECharts} Javascript library within a JSF component.
The source code for the component is given below.
\lstinputlisting[language=Java]{../helloworld-primefaces/taglib/src/main/java/be/e_contract/jsf/taglib/EChartsComponent.java}

The source code for the corresponding renderer is given below.
\lstinputlisting[language=Java]{../helloworld-primefaces/taglib/src/main/java/be/e_contract/jsf/taglib/EChartsRenderer.java}
The value is not directly passed during \texttt{encodeBegin}.
Instead we let the widget fetch the value via an AJAX call.
Within the decode method we pass the component value towards the client-side via \texttt{addCallbackParam}.

The source code of the Javascript resource \texttt{echarts-widget.js} is given below.
\lstinputlisting[language=Javascript]{../helloworld-primefaces/taglib/src/main/resources/META-INF/resources/ejsf/echarts-widget.js}
Here we perform an AJAX call to retrieve our component value, and use this to configure the echarts Javascript instance.

We register this custom component within our \texttt{.taglib.xml} tag library descriptor.
\lstinputlisting[language=XML,linerange=echarts]{../helloworld-primefaces/taglib/src/main/resources/META-INF/ejsf.taglib.xml}

An example usage is demonstrated below:
\lstinputlisting[language=XML]{../helloworld-primefaces/war/src/main/webapp/echarts.xhtml}

With the CDI controller as follows:
\lstinputlisting[language=Java]{../helloworld-primefaces/war/src/main/java/be/e_contract/jsf/helloworld/EChartsController.java}

To bundle the Javascript for Apache EChart within our JSF library, we use some specific build instructions as indicated in the following \texttt{pom.xml}.
\lstinputlisting[language=XML]{../helloworld-primefaces/taglib/pom.xml}
With the \texttt{src/main/node/package.json} file containing:
\lstinputlisting[language=XML]{../helloworld-primefaces/taglib/src/main/node/package.json}
Hence we use \texttt{node} and \texttt{npm} to retrieve the ECharts Javascript package.

Via the \texttt{maven-resources-plugin} we selectively pick what we need out of the downloaded \texttt{echarts} node module as was retrieved by the \texttt{frontend-maven-plugin}.