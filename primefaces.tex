% Enterprise JSF project.
%
% Copyright 2022-2023 e-Contract.be BV. All rights reserved.
% e-Contract.be BV proprietary/confidential. Use is subject to license terms.

\chapter{PrimeFaces}

JSF has known a lot of component libraries over its history.
After almost two decades of JSF however, natural selection did set in,
resulting in the survival of just a few really high quality JSF component libraries.
The most prominent here is PrimeFaces \cite{PrimeFaces}.
Some of its major features:
\begin{itemize}
	\item Lots of components well beyond standard JSF component set.
	\item Client-side API for the components.
	\item Dialog framework.
	\item Configurable style.
\end{itemize}

\section{Introduction}
Converting our \ref{sec:basic-input-output} \nameref{sec:basic-input-output} example from page \pageref{sec:basic-input-output} to PrimeFaces gives us the following.
\lstinputlisting[language=XML]{helloworld-primefaces/war/src/main/webapp/hello-world.xhtml}
Most visible change here is that they renamed the \texttt{render} attribute to the more intuitive \texttt{update} attribute.

\section{CRUD with Dialogs}
Within this section, we demonstrate the dialog framework provided by PrimeFaces.
Our JSF example page looks as follows:
\lstinputlisting[language=XML]{helloworld-primefaces/war/src/main/webapp/crud-dialogs.xhtml}
Notice how to invoke client-side API of certain widgets via the \texttt{wdigetVar} attributes to give them a name, and via the \texttt{PF} function to locate these widgets.

With our Javascript resource \texttt{crud-dialogs.js} containing:
\lstinputlisting[language=Javascript]{helloworld-primefaces/war/src/main/webapp/resources/js/crud-dialogs.js}
Notice here the \texttt{itemAdded} argument that we set within the controller below.

And the CDI controller as follows:
\lstinputlisting[language=Java]{helloworld-primefaces/war/src/main/java/be/e_contract/jsf/helloworld/CRUDController.java}

\section{Close Dialog Client Behavior}
Let us start with a simple custom client behavior to close the dialogs.
\lstinputlisting[language=Java]{helloworld-primefaces/taglib/src/main/java/be/e_contract/jsf/taglib/CloseDialogClientBehavior.java}
Here we simply locate the dialog in which the \texttt{p:commandButton} sits.
Once we have the dialog, we use its \texttt{widgetVar} to be able to close it at the client-side.

We register this custom component within our \texttt{.taglib.xml} tag library descriptor.
\lstinputlisting[language=XML,linerange=closeDialog]{helloworld-primefaces/taglib/src/main/resources/META-INF/ejsf.taglib.xml}

We can now use this as follows within the PrimeFaces dialogs:
\lstinputlisting[language=XML]{helloworld-primefaces/war/src/main/webapp/close-dialog.xhtml}


\section{Widgets}
Our custom component with corresponding client-side widget is composed out of the actual component, and the corresponding renderer.
The source code for the component is given below.
\lstinputlisting[language=Java]{helloworld-primefaces/taglib/src/main/java/be/e_contract/jsf/taglib/ExampleComponent.java}
The component simply defines a \texttt{widgetVar} attribute that is used by our custom renderer.

The source code for the corresponding renderer is given below.
\lstinputlisting[language=Java]{helloworld-primefaces/taglib/src/main/java/be/e_contract/jsf/taglib/ExampleRenderer.java}
Using the PrimeFaces \texttt{WidgetBuilder} we bind our custom (client-side) widget to our component.

The source code of the Javascript resource \texttt{example-widget.js} is given below.
\lstinputlisting[language=Javascript]{helloworld-primefaces/taglib/src/main/resources/META-INF/resources/ejsf/example-widget.js}
Notice how we extend the PrimeFaces \texttt{BaseWidget}. Via the \texttt{init} method our custom widget gets initialized by the PrimeFaces client-side framework.

We register this custom component within our \texttt{.taglib.xml} tag library descriptor.
\lstinputlisting[language=XML,linerange=exampleComponent]{helloworld-primefaces/taglib/src/main/resources/META-INF/ejsf.taglib.xml}

An example usage is demonstrated below:
\lstinputlisting[language=XML]{helloworld-primefaces/war/src/main/webapp/widget.xhtml}

\section{Turning a Javascript library into a JSF component}
Within this section we will demonstrate how to turn a Javascript library into an easy to use JSF component.
As example we will wrap the Apache ECharts \cite{ECharts} Javascript library within a JSF component.
The source code for the component is given below.
\lstinputlisting[language=Java]{helloworld-primefaces/taglib/src/main/java/be/e_contract/jsf/taglib/EChartsComponent.java}

The source code for the corresponding renderer is given below.
\lstinputlisting[language=Java]{helloworld-primefaces/taglib/src/main/java/be/e_contract/jsf/taglib/EChartsRenderer.java}

The source code of the Javascript resource \texttt{echarts-widget.js} is given below.
\lstinputlisting[language=Javascript]{helloworld-primefaces/taglib/src/main/resources/META-INF/resources/ejsf/echarts-widget.js}

We register this custom component within our \texttt{.taglib.xml} tag library descriptor.
\lstinputlisting[language=XML,linerange=echarts]{helloworld-primefaces/taglib/src/main/resources/META-INF/ejsf.taglib.xml}

An example usage is demonstrated below:
\lstinputlisting[language=XML]{helloworld-primefaces/war/src/main/webapp/echarts.xhtml}

And the CDI controller as follows:
\lstinputlisting[language=Java]{helloworld-primefaces/war/src/main/java/be/e_contract/jsf/helloworld/EChartsController.java}

To retrieve the Javascript for Apache EChart, we use some specific build instructions as indicated in the following \texttt{pom.xml}.
\lstinputlisting[language=XML]{helloworld-primefaces/taglib/pom.xml}